\documentclass{article}
\usepackage[utf8]{inputenc}
\usepackage{amsmath}
\usepackage[spanish]{babel}
\title{Apuntes de programación lineal}
\author{Jesús}
\begin{document}

\maketitle
\section{Introduccion}
La forma estandar de un problema de programación lineal es:
Dados una matriz $A$ y vectores
$b,c$ maximizar $c^Tx$ sujeto a $Ax\leq b$.

\medskip
Existe un método para resolver este tipo de problemas, llamado el
método simplex,tal método requiere de la forma simplex o forma
ecuacionaria, la cual es:
Dados una matriz $A$ y vectores $b,c$ maximizar $c^Tx$ sujeto a $Ax =
b$.

\medskip
Usualmente es necesario añadir variables para cumplir con la igualdad
(variables de holgura).
\section{Tablas}
\begin{tabular}{|c|c|c|}
  \hline
  &A&B\\
  \hline
  Maquina1&1&2\\
  \hline
  Maquina2&1&1\\
  \hline   
\end{tabular}
\section{Matrices}
\begin{equation}
  \label{eq:1}
  A=
  \begin{pmatrix} 
    0&-9&0\\
    5&0&1\\
    10&1&1\\
  \end{pmatrix}
  \begin{pmatrix}
     0&-9&0\\
    5&0&1\\
    10&1&1\\
  \end{pmatrix}
\end{equation}
\end{document}






                                         